\documentclass{fdclreport}
\usepackage{lipsum}     % dummy text
\usepackage{cite}       % citation
\usepackage{pgfgantt}   % gantt chart
\usepackage[utf8]{inputenc}
\usepackage{graphicx}
\usepackage{amsmath}
\usepackage{amsfonts}
\usepackage[version=4]{mhchem}
\usepackage{siunitx}
\usepackage{array}
\usepackage{longtable}%tabulary
\usepackage{subfigure}
\usepackage{multirow}
\usepackage{makecell}
\usepackage{bm}

\project{TD-21}
\title{Note for Journal}
\author{Yeontaek Jung}
\recipient{Prof. Youdan Kim}
\date{\today}
\reportnumber{0}

\begin{document}
\maketitle
\section{Idea}


\section{Contributions}
\begin{itemize}
	\item Guidance command generated by the proposed algorithm at start is relatively smaller than that of ZEM algorithm
	\item PEG algorithm is modified to be used for solid propellant anti-ballistic missile
	\item Instability of PEG algorithm caused as time-to-go goes to zero is handled
	\item The proposed algorithm does not need to fixed PIP because PIP is updated in real time
	\item The proposed algorithm dose not have iteration step
	\item The proposed algorithm is robust to the uncertainty of thrust
\end{itemize}

\section{Issues}
\begin{itemize}
	\item 참고문헌을 잘 찾았는지 확신이 서지 않는다. 키워드 확보가 필요
\end{itemize}

\section{To Do}

\makedetailedcontents

\section{Equations}

\subsection{Dynamic Equations in Exo-atmosphere}
Aerodynamic force can be neglected in the exo-atmosphere, and therefore the dynamics of the target and DM can be modeled as a point mass with three degree of freedom in the exo-atmosphere.
In this study, the notation defined in~\cite{stevens} is used.
Subscript $O$ denotes the center of the Earth modeled as a sphere, $M$ denotes a DM, $T$ denotes a target, and $i$ denotes the Earth-centered inertial (ECI) frame or coordinate.
Bold letters denote vectors and thin letters are scalars.

Dynamic equations of DM can be represented as follows.
\begin{align}
	^i\dot{\bm{r}}_{M/O} &= \bm{v}_{M/i} \label{eq:r}\\
	^i\dot{\bm{v}}_{M/i} &= \frac{T}{m} \bm{\hat{u}} + \bm{g}(\bm{r}_{M/O}) \label{eq:v}\\
	\dot{m} &= -\frac{T}{g_0 I_{sp}} \label{eq:m}
\end{align}
where $\bm{r}_{M/O}$ is the position vector of the DM with respect to the center of the Earth, $\bm{v}_{M/i}$ is the velocity vector of the DM in the ECI frame, $T$ is a thrust of the DM, $m$ is a mass of the DM, $g_0$ is a gravitational acceleration of the Earth at sea level, $I_{sp}$ is a specific impulse at vacuum, and $\bm{\hat{u}}$ is the unit direction vector of thrust.
The dynamic model can be rewritten as follows.
\begin{align}
	\dot{\bm{x}} &= \bm{f(x, u)} \label{eq:dyn} 
\end{align}
where $\bm{x} = [\bm{r}_{M/O}^T \quad \bm{v}_{M/i}^T \quad m]^T$.
Dynamic equations of the target can be represented as follows.
\begin{align}
	^i\dot{\bm{r}}_{T/O} &= \bm{v}_{T/i} \label{eq:rT} \\
	^i\dot{\bm{v}}_{T/i} &= \bm{g}(\bm{r}_{T/O}) \label{eq:vT}
\end{align}
where $\bm{r}_{T/O}$ is the position vector of the target with respect to the center of Earth, and $\bm{v}_{T/i}$ is the velocity vector of the target in the ECI frame.
In this study, the following inverse-square gravitation model is used.
\begin{align}
	\bm{g}(\bm{r}) &= -\frac{\mu_e}{||\bm{r}||^3}\bm{r}
\end{align}
where $\mu_e$ is the gravitational constant of the Earth.

\subsection{ZEM}


\subsection{Optimal Control Problem of Iterative Guidance Mode} \label{sec:optimal}

\begin{align}
	\min_{\bm{u}} J &= \int_{0}^{t_{f}} dt \label{eq:J}
\end{align}
subject to
\begin{align}
	\dot{\bm{x}} &= \bm{f(x, u)} \label{eq:ode} \\ 
	\bm{\hat{u}}^T\bm{\hat{u}} &= 1
\end{align}
and
\begin{align}
	\bm{r}_{M/O}(t_f) = \bm{r}_{M/O, des},& \quad \bm{v}_{M/i}(t_f) = \bm{v}_{M/i, des}  \label{eq:xd}
\end{align}
where $\bm{r}_{M/O, def}$ and $\bm{v}_{M/O, des}$ are the desired position and velocity vector of the DM at end of the flight, respectively.
State constraint (\ref{eq:xd}) at the end of the flight is the desired location and the desired velocity at the location for a launch vehicle to enter the desired orbit.
From here, the subscript of the DM is omitted for the simplicity of notation.

Let us define a Hamiltonian $H$ as follows.
\begin{align}
	H &= -1 + \bm{\lambda_r}^T \bm{v} + \bm{\lambda_v}^T (\frac{T}{m}\bm{u} + \bm{g}(\bm{r})) -\lambda_m \frac{T}{g_0 I_{sp}} + \lambda_u (1-\bm{\hat{u}}^T \bm{\hat{u}}).
\end{align}
Costate equations and necessary condition can be represented as follows.
\begin{align}
	\frac{d \bm{\lambda_r}}{dt} &= - \bm{\lambda_v}^T \frac{\partial \bm{g}(\bm{r})}{\partial \bm{r}} \\
	\frac{d \bm{\lambda_v}}{dt} &= -\bm{\lambda_r} \\ 
	\frac{\partial H}{\partial \bm{u}} &= \frac{T}{m} \bm{\lambda_v} - \lambda_u \bm{\hat{u}} = 0.
\end{align}
As a result, the optimal control input is obtained as follows.
\begin{align}
	\bm{\hat{u}} &= \frac{T}{m \lambda_u} \bm{\lambda_v}.
\end{align}
Note that the optimal control input $\bm{u}$ is parallel to the costate vector $\bm{\lambda_v}$.

\textbf{\emph{Assumption 1}}: Gravity acceleration $\bm{g}(\bm{r})$ is constant.

Under Assumption 1, $\bm{\lambda_r}$ is constant and $\bm{\lambda_v}$ is a linear function of time, which is expressed as follows.
\begin{align}
	\bm{\lambda_v} &= -\bm{a} t + \bm{b}
\end{align}
where $\bm{a}$ and $\bm{b}$ are constant vectors.

\subsection{PEG}
Design $\bm{\lambda_v}$ as follows.
\begin{align}
	\bm{\lambda_v} &= \bm{\hat{\lambda}} + \bm{\dot{\lambda}} (t-t_{\lambda})
\end{align}
such that
\begin{align}
	\bm{\hat{\lambda}}^T \bm{\dot{\lambda}} &= 0
\end{align}
Then,
\begin{align}
	\bm{\hat{u}} &= \frac{\bm{\hat{\lambda}} + \bm{\dot{\lambda}} (t-t_{\lambda})}{|\bm{\hat{\lambda}} + \bm{\dot{\lambda}|}}
	= \frac{\bm{\hat{\lambda}} + \bm{\dot{\lambda}} (t-t_{\lambda})}{\sqrt{1 + \bm{\dot{\lambda}}^T\bm{\dot{\lambda}}(t-t_{\lambda})^2}}
\end{align}
Under assumption that $||\bm{\dot{\lambda}}||_2$ is small,
\begin{align}
	\bm{\hat{u}} & \approx \bm{\hat{\lambda}} + \bm{\dot{\lambda}}(t-t_{\lambda}) \label{eq:approx}
\end{align}

Substituting (\ref{eq:approx}) into (\ref{eq:v}) and integrating.
\begin{align}
	\bm{v}(t_{go}) &= \bm{v}(0) + L\bm{\hat{\lambda}} + \bm{\dot{\lambda}}(J - L t_{\lambda}) + \bm{v}_{grav} \\
	\bm{r}(t_{go}) &= \bm{r}(0) + \bm{v}(0)t_{go} + S \bm{\hat{\lambda}} + \bm{\dot{\lambda}}(Q - S t_{\lambda}) + \bm{r}_{grav} \\
	\bm{v}_{grav} &= \int_{0}^{t_{go}}\bm{g}(\bm{r}) dt \\
	\bm{r}_{grav} &= \int_{0}^{t_{go}} \int_{0}^{\tau} \bm{g}(\bm{r}) dsd\tau
\end{align}
$L$, $J$, $S$, and $Q$ are thrust integrating factors as follows.
\begin{align}
	L &= \int_{0}^{t_{go}} \frac{F}{m}dt = V_{ex} \log \frac{\tau}{\tau-t_{go}} \\
	J &= \int_{0}^{t_{go}} \frac{F}{m}t dt = V_{ex} \left( \tau \log \frac{\tau}{\tau - t_{go}} - t_{go} \right) \\
	S &= \int_{0}^{t_{go}} \int_{0}^{\tau} \frac{F}{m}dsd\tau = -V_{ex} \left[ \left(\tau - t_b \right) \log \frac{\tau}{\tau-t_{go}} - t_{go} \right] \\
	Q &= \int_{0}^{t_{go}} \int_{0}^{\tau} \frac{F}{m}sdsd\tau = -V_{ex} \left[ \frac{t_{go}^2}{2} + \tau \left\{ \left(\tau - t_{go}\right) \log \frac{\tau}{\tau - t_{go}} - t_{go} \right\} \right] \\
	\tau &= -\frac{m_0}{\dot{m}}
\end{align}
To make $\bm{v}(t_{go}) = \bm{v}_{d}$ and $\bm{r}(t_{go}) = \bm{r}_{d}$, let us define $\bm{v}_{go}$ and $\bm{r}_{go}$ as follows.
\begin{align}
	\bm{v}_{go} &= \bm{v}_{d} - (\bm{v}(0) + \bm{v}_{grav}) = L\bm{\hat{\lambda}} + \bm{\dot{\lambda}}(J - L t_{\lambda}) \\
	\bm{r}_{go} &= \bm{r}_{d} - (\bm{r}(0) + \bm{v}(0)t_{go} + \bm{r}_{grav}) = S \bm{\hat{\lambda}} + \bm{\dot{\lambda}}(Q - S t_{\lambda})
\end{align}
Then,
\begin{align}
	\bm{\hat{\lambda}} &= \frac{\bm{v}_{go}}{L} \\
	\bm{\dot{\lambda}} &= \frac{\bm{r}_{go} - S \bm{\hat{\lambda}}} {Q - S t_{\lambda}} \\
	t_{\lambda} &= \frac{J}{L}
\end{align}

\subsection{Position PEG}
Only consider final position constraints. Then, $\bm{\lambda_v}(t_f) = 0$. Thus,
\begin{align}
	\bm{\lambda_v} &= (t - t_{f})\bm{a}
\end{align}
It means that thrust direction is constant during boost phase.
\begin{align}
	\bm{\hat{u}} &= \bm{\hat{\lambda}} \label{eq:constant}
\end{align}
Substituting (\ref{eq:constant}) into (\ref{eq:v}) and integrating with $t_{go}=t_b$.
\begin{align}
	\bm{v}_b &= \bm{v}(t_b) = \bm{v}(0) + L \bm{\hat{\lambda}} + \bm{v}_{grav} \label{eq:v_b} \\ 
	\bm{r}_b &= \bm{r}(t_b) = \bm{r}(0) + \bm{v}(0)t_b + S \bm{\hat{\lambda}} + \bm{r}_{grav} \label{eq:r_b}\\ 
	\bm{v}_{T,b} &= \bm{v}_T(t_b) = \bm{v}_T(0) + \bm{g}_T t_b \label{eq:v_T_b} \\ 
	\bm{r}_{T,b} &= \bm{r}_T(t_b) = \bm{r}_T(0) + \bm{v}_T(0) t_b + \frac{1}{2}\bm{g}_T t_b^2 \label{eq:r_T_b}
\end{align}
Using (\ref{eq:v_b})-(\ref{eq:r_T_b}), ZEM can be computed.
\begin{align}
	\bm{r}_{ZEM} &= \bm{f}(\bm{r}_b, \bm{v}_b, \bm{v}_{T,b}, \bm{r}_{T,b})
\end{align}
We want to update $\bm{\hat{\lambda}}$ to make $\bm{r}_{ZEM}$ be zero.
\begin{align}
	\bm{\hat{\lambda}}^{k+1} = \bm{\hat{\lambda}}^k + \bm{\Delta \lambda} \label{eq:update}
\end{align}
Substituting (\ref{eq:update}) into (\ref{eq:v}) and integrating with $t_{go}=t_b$.
\begin{align}
\begin{split}
	\bm{r}_b^{k+1} &= \int_{0}^{t_b} \int_{0}^{\tau} \frac{F}{m} \bm{\hat{\lambda}}^{k+1} + \bm{g}(\bm{r}) ds d\tau \\
								 &= \int_{0}^{t_b} \int_{0}^{\tau} \frac{F}{m} \bm{\hat{\lambda}}^{k} + \bm{g}(\bm{r}) + \frac{F}{m} \bm{\Delta \lambda} ds d\tau \\
								 &= \bm{r}_b^{k} + S \bm{\Delta \lambda}
\end{split}
\end{align}
To make $\bm{r}_{ZEM} = 0$, let us define optimal problem as follows.
\begin{align}
	\begin{split}
	\min_{\bm{\Delta \lambda}} \quad & \left( S \bm{\Delta \lambda} - \bm{r}_{ZEM} \right)^T \left( S \bm{\Delta \lambda} - \bm{r}_{ZEM} \right) \\
	\textrm{s.t.} \quad & (\bm{\hat{\lambda}}^k + \bm{\Delta \lambda})^T (\bm{\hat{\lambda}}^k + \bm{\Delta \lambda}) = 1 \label{eq:opt}
\end{split}
\end{align}
If $\bm{\Delta \lambda} = \bm{\hat{\lambda}}^{k+1} - \bm{\hat{\lambda}}^{k}$ is substituted into (\ref{eq:opt}), then the optimization problem is changed as follows.
\begin{align}
	\begin{split}
	\min_{\bm{\hat\lambda}^{k+1}} \quad & \left( S \bm{\hat{\lambda}}^{k+1} - (S \bm{\hat{\lambda}}^{k} + \bm{r}_{ZEM}) \right)^T \left( S \bm{\hat{\lambda}}^{k+1} - (S \bm{\hat{\lambda}}^{k} + \bm{r}_{ZEM}) \right) \\
	\textrm{s.t.} \quad & \bm{\hat{\lambda}}^{{k+1}^T} \bm{\hat{\lambda}}^{k+1} = 1 \label{eq:opt2}
\end{split}
\end{align}
The solution of the optimization problem is as follows.
\begin{align}
	\bm{\hat{\lambda}}^{k+1} &= \frac{\bm{\hat{\lambda}}^k + \frac{1}{S} \bm{r}_{ZEM}}{||\bm{\hat{\lambda}}^k + \frac{1}{S} \bm{r}_{ZEM}||_2}
\end{align}
$S$ goes to zero when $t_b$ goes to zero. Thus, the update law can be practically used as follows.
\begin{align}
	\bm{\hat{\lambda}}^{k+1} &= \frac{\bm{\hat{\lambda}}^{k} + K \bm{r}_{ZEM}/R_{normalize}}{||\bm{\hat{\lambda}}^{k} + K \bm{r}_{ZEM}/R_{normalize}||_2}
\end{align}
This algorithm does not need iteration step.

\subsection{For the robustness to the thrust uncertainty}
For the robustness to the thrust uncertainty, $\bm{r}_{ZEM}$ can be designed as follows.
\begin{align}
	\bm{r}_{ZEM}(t_b) &= \bm{f}(\bm{r}_b, \bm{v}_b, \bm{v}_{T,b}, \bm{r}_{T,b}) \\
	\bm{r}_{ZEM}(t_0) &= \bm{f}(\bm{r}(t_0), \bm{v}(t_0), \bm{v}_T(t_0), \bm{r}_T(t_0)) \\
	\bm{r}_{ZEM} &= (1-\rho) \bm{r}_{ZEM}(t_b) + \rho \bm{r}_{ZEM}(t_0) 
\end{align}
$\rho$ can be designed as follows.
\begin{align}
	\rho &= 1 - \frac{t_b}{t_{b0}} \\
	\rho &= log \left( (1-e)\frac{t_b}{t_{b0}} + e \right)
\end{align}
$t_b$ is remaining burnout time and $t_{b0}$ is initial burnout time.

\subsection{Position-Velocity PEG}
Velocity constraint is considered, too.
\begin{align}
	\bm{\hat{u}} &\approx \bm{\hat{\lambda}} + \bm{\dot{\lambda}}(t-t_{\lambda}) \\
\begin{split}
	\bm{v}_b^{k+1} &= \int_{0}^{t_b} \frac{F}{m} (\bm{\hat{\lambda}}^{k+1} + \bm{\dot{\lambda}}^{k+1}(t-t_{\lambda}) )+ \bm{g}(\bm{r}) dt \\
								 &= \int_{0}^{t_b} \frac{F}{m} (\bm{\hat{\lambda}}^{k} + \bm{\dot{\lambda}}^{k}(t-t_{\lambda}) )+ \bm{g}(\bm{r}) dt	+ \int_{0}^{t_b} \frac{F}{m} (\bm{\Delta \hat{\lambda}} + \bm{\Delta \dot{\lambda}}(t-t_{\lambda}))dt\\
								 &= \bm{v}_b^{k} + L \bm{\Delta \hat{\lambda}} + \bm{\Delta \dot{\lambda}} (J - L t_{\lambda})
\end{split} \\
\begin{split}
	\bm{r}_b^{k+1} &= \int_{0}^{t_b} \int_{0}^{\tau} \frac{F}{m} (\bm{\hat{\lambda}}^{k+1} + \bm{\dot{\lambda}}^{k+1}(s-t_{\lambda})) + \bm{g}(\bm{r}) ds d\tau \\
								 &= \int_{0}^{t_b} \int_{0}^{\tau} \frac{F}{m} (\bm{\hat{\lambda}}^{k} + \bm{\dot{\lambda}}^{k}(s-t_{\lambda}) ) + \bm{g}(\bm{r}) ds d\tau + \int_{0}^{t_b} \int_{0}^{\tau} \frac{F}{m} (\bm{\Delta \hat{\lambda}} + \bm{\Delta \dot{\lambda}}(s-t_{\lambda}))ds d\tau\\
								 &= \bm{r}_b^{k} + S \bm{\Delta \hat{\lambda}} + \bm{\Delta \dot{\lambda}} (Q-St_{\lambda})
\end{split}
\end{align}
Let
\begin{align}
	t_{\lambda} &= \frac{Q}{S}
\end{align}
then
\begin{align}
	\bm{\hat{\lambda}}^{k+1} &= \frac{\bm{\hat{\lambda}}^{k} + K_{hat} \bm{r}_{ZEM}/R_{normalize}}{||\bm{\hat{\lambda}}^{k} + K_{hat} \bm{r}_{ZEM}/R_{normalize}||_2} \\
	\bm{r}_{ZEM}(t_b) &= \bm{f}(\bm{r}_b, \bm{v}_b, \bm{v}_{T,b}, \bm{r}_{T,b}) \\
	\bm{r}_{ZEM}(t_0) &= \bm{f}(\bm{r}(t_0), \bm{v}(t_0), \bm{v}_T(t_0), \bm{r}_T(t_0)) \\
	\bm{r}_{ZEM} &= (1-\rho) \bm{r}_{ZEM}(t_b) + \rho \bm{r}_{ZEM}(t_0) \\
	\bm{\Delta \hat{\lambda}} &= \bm{\hat{\lambda}}^{k+1} - \bm{\hat{\lambda}}^k
\end{align}
$\bm{v}_d$ is opposite direction to the velocity of target as follows.
\begin{align}
	\bm{\hat{v}}_d &= \frac{\bm{v}_{T}(t_{go})}{||\bm{v}_{T}(t_{go})||_2}
\end{align}
Then, $\bm{\Delta \dot{\lambda}}$ can be represented as follows.
\begin{align}
\begin{split}
	\bm{v}(t_{go})^{k+1} &= \bm{v}(t_{b})^{k+1} + (t_{go} - t_b)\bm{g} \\
											 &= \bm{v}(t_{go})^{k} + L \bm{\Delta \hat{\lambda}} + \bm{\Delta \dot{\lambda}} (J - L t_{\lambda}) \\
											 &= \bm{v}_d
\end{split}\\
\begin{split}
	(J-Lt_{\lambda}) \bm{\Delta \dot{\lambda}} &= \bm{v}_d - \bm{v}(t_{go})^k - L \bm{\Delta \hat{\lambda}} \\
																						 &= \bm{v}_{ZEM} - L \bm{\Delta \hat{\lambda}}
\end{split} \\
	\bm{\dot{\lambda}}^{k+1} &= \bm{\dot{\lambda}}^{k} + K_{dot} (\bm{v}_{ZEM} - L \bm{\Delta \hat{\lambda}} / V_{normalize})
\end{align}
For robustness,
\begin{align}
	\bm{v}_{ZEM}(t_b) &= (\bm{v}_T(t_0) + \bm{g}_T * (t_{go} - t_0)) - (\bm{v}(t_0) + L \bm{\hat{\lambda}}^{k} + (J-L t_{\lambda}) \bm{\dot{\lambda}}^{k} + \bm{g}_M (t_{go} - t_b)) \\
	\bm{v}_{ZEM}(t_0) &= (\bm{v}_T(t_0) + \bm{g}_T * (t_{go} - t_0)) - (\bm{v}(t_0) + \bm{g}_M (t_{go} - t_0)) \\
	\bm{v}_{ZEM} &= (1-\rho) \bm{v}_{ZEM}(t_b) + \rho \bm{v}_{ZEM}(t_0)
\end{align}

\subsection{Enhance robustness for miss distance}
Robustness for miss distance is enhanced when gain for $\bm{\Delta \hat{\lambda}}$ is bigger and gain for $\bm{\Delta \dot{\lambda}}$ is smaller.
\begin{align}
	\bm{\hat{\lambda}}^{k+1} &= \frac{\bm{\hat{\lambda}}^{k} + K_{hat} exp(K_{1}(1-t_b/t_{b0})) \bm{r}_{ZEM}/R_{normalize}}{||\bm{\hat{\lambda}}^{k} + K_{hat} exp(K_{1}(1-t_b/t_{b0})) \bm{r}_{ZEM}/R_{normalize}||_2} \\
	\bm{\dot{\lambda}}^{k+1} &= \bm{\dot{\lambda}}^{k} + K_{dot} exp(K_{2}(t_b/t_{b0}-1)) (\bm{v}_{ZEM} - L \bm{\Delta \hat{\lambda}} / V_{normalize})
\end{align}

\section{Low stage}
\begin{align}
	\bm{v}_I(t) &= \begin{cases}
		\bm{v}_I(0) + \int_0^t \frac{F}{m}\bm{\hat{u}} + \bm{g}(\bm{r}_I) ds, & 0 \leq t < t_{b,1}\\
		\bm{v}_I(t_{b,1}) + \int_{t_{b, 1}}^t \bm{g}(\bm{r}_I) ds, & t_{b,1} \leq t < t_{b,1} + t_c \\
		\bm{v}_I(t_{b,1} + t_c) + \int_{t_{b, 1} + t_c}^t \frac{F}{m}\bm{\hat{u}} + \bm{g}(\bm{r}_I) ds, & t_{b,1} + t_c \leq t < t_{b,1} + t_c + t_{b,2}\\
		\bm{v}_I(t_{b,1} + t_c + t_{b,2}) + \int_{t_{b, 1} + t_c + t_{b, 2}}^t \bm{g}(\bm{r}_I) ds, & t_{b,1} + t_c + t_{b, 2} \leq t
		\end{cases}
\end{align}

\begin{align}
	\bm{r}_I(t) &= \begin{array}{ll} \bm{r}_{grav}(t) + \left [ S_1 + S_2 + L_1 (t - t_{b,1}) + L_2 (t - t_{b, 1} - t_c - t_{b, 2}) \right ] \bm{\hat{u}}, & t_{b, 1} + t_c + t_{b, 2} \leq t
	\end{array}
\end{align}

\begin{align}
	\bm{r}_{grav}(t) &= \bm{r}(0) + \bm{v}(0) t + \int_0^t \int_0^{\tau} \bm{g}(\bm{r}_I) ds d\tau
\end{align}

\begin{align}
	L_1 &= \int_0^{t_{b, 1}} \frac{F}{m}ds \\
	L_2 &= \int_{t_{b, 1} + t_c}^{t_{b, 1} + t_c + t_{b, 2}} \frac{F}{m}ds \\
	S_1 &= \int_0^{t_{b, 1}} \int_0^{\tau} \frac{F}{m}dsd\tau \\
	S_2 &= \int_{t_{b, 1} + t_c}^{t_{b, 1} + t_c + t_{b, 2}} \int_{t_{b, 1}+t_c}^{\tau} \frac{F}{m}dsd\tau
\end{align}

\begin{align}
	\bm{r}_T(t_{go}) &= \bm{r}_{grav}(t_{go}) + \left [ (L_1 + L_2) t_{go} + S_1 + S_2 - L_1 t_{b, 1} - L_2 (t_{b, 1} + t_c + t_{b, 2}) \right ] \bm{\hat{u}} \\
									 & \left [ \bm{r}_T(t_{go}) - \bm{r}_{grav}(t_{go}) \right ]^T \left [ \bm{r}_T(t_{go}) - \bm{r}_{grav}(t_{go}) \right ] = \left [ (L_1 + L_2) t_{go} + S_1 + S_2 - L_1 t_{b, 1} - L_2 (t_{b, 1} + t_c + t_{b, 2}) \right ]^2
\end{align}

\begin{align}
	A_4 t_{go}^4 + A_3 t_{go}^3 + A_2 t_{go}^2 + A_1 t_{go} + A_0 = 0
\end{align}

\begin{align}
	A_4 &= \frac{1}{4} \Delta \bm{g}^T \Delta \bm{g} \\
	A_3 &= \Delta \bm{g}^T \Delta \bm{v}_0 \\
	A_2 &= \Delta \bm{g}^T \Delta \bm{r}_0 + \Delta \bm{v}_0^T\Delta \bm{v}_0 \\
	A_1 &= 2 \Delta \bm{v}_0^T \Delta \bm{r}_0 \\
	A_0 &= \Delta \bm{r}_0^T \Delta \bm{r}_0
\end{align}

\begin{align}
	\Delta \bm{g} &= \bm{g}(\bm{r}_T(0)) - \bm{g}(\bm{r}_I(0)) \\
	\Delta \bm{v}_0 &= \bm{v}_T(0) - \bm{v}_I(0) \\
	\Delta \bm{2}_0 &= \bm{2}_T(0) - \bm{2}_I(0)
\end{align}

\newpage
\section{Guidance Law}
\subsection{Consider position and velocity constraints in both stage 2 and 3}
\begin{align}
	\bm{\hat{u}}(t) &= \begin{cases}
		\bm{\hat{\lambda}} + \bm{\dot{\lambda}}(t-t_{\lambda}), & 0 \leq t < t_{b,1}\\
		0, & t_{b,1} \leq t < t_{b,1} + t_c \\
		\bm{\hat{\lambda}} + \bm{\dot{\lambda}}(t-t_{\lambda}), & t_{b,1} + t_c \leq t < t_{b,1} + t_c + t_{b,2}\\
		0, & t_{b,1} + t_c + t_{b, 2} \leq t
		\end{cases}
\end{align}

\begin{align}
	\bm{v}_I(t_{go}) &= \bm{v}_I(0) + \int_{0}^{t} \bm{g}(\bm{r}_I) d\tau + L \bm{\hat{\lambda}} + \bm{\dot{\lambda}} (J - t_{\lambda} L) \\
	\begin{split}
	\bm{r}_I(t_{go}) &= \bm{r}_I(0) + \int_{0}^{t} \bm{v}_I(0) ds + \int_{0}^{t}\int_{0}^{s} \bm{g}(\bm{r}) d\tau ds \\
									 & + (S + L_1 (t_c + t_{b,2} + t_T) + L_2 t_T) \bm{\hat{\lambda}} \\
									 & + (Q + J_1 (t_c + t_{b,2} + t_T) + J_2 t_T - t_{\lambda} (S + L_1 (t_c + t_{b,2} + t_T) + L_2 t_T)) \bm{\dot{\lambda}}
\end{split}
\end{align}


\subsection{Consider position and velocity constraints in stage 2 and consider only position constraint in stage 3}
\begin{align}
	\bm{\hat{u}}(t) &= \begin{cases}
		\bm{\hat{\lambda}} + \bm{\dot{\lambda}}(t-t_{\lambda}), & 0 \leq t < t_{b,1}\\
		0, & t_{b,1} \leq t < t_{b,1} + t_c \\
		\bm{\hat{\lambda}}, & t_{b,1} + t_c \leq t < t_{b,1} + t_c + t_{b,2}\\
		0, & t_{b,1} + t_c + t_{b, 2} \leq t
		\end{cases}
\end{align}

\begin{align}
	\bm{v}_I(t_{go}) &= \bm{v}_I(0) + \int_{0}^{t} \bm{g}(\bm{r}_I) d\tau + L \bm{\hat{\lambda}} + \bm{\dot{\lambda}} (J_1 - t_{\lambda} L_1) \\
	\begin{split}
	\bm{r}_I(t_{go}) &= \bm{r}_I(0) + \int_{0}^{t} \bm{v}_I(0) ds + \int_{0}^{t}\int_{0}^{s} \bm{g}(\bm{r}) d\tau ds \\
									 & + (S + L_1 (t_c + t_{b,2} + t_T) + L_2 t_T) \bm{\hat{\lambda}} \\
									 & + (Q_1 + J_1 (t_c + t_{b,2} + t_T) - t_{\lambda} (S_1 + L_1 (t_c + t_{b,2} + t_T))) \bm{\dot{\lambda}}
\end{split}
\end{align}



	












%%%%%%%%%%%%%%%%%%%%%%%%%%%%%%%%%%%%%%%%%%%%%%%%%
\newpage
\bibliographystyle{ieeetr}
\bibliography{references}
\end{document}
